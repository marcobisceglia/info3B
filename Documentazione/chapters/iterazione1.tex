\chapter{Iterazione 1}

\section{Introduzione}
Nella prima iterazione si è scelto di implementare i seguenti casi d'uso:

\begin{itemize}
    \item UC1: Gestione barche \textit{[Astratto]}
          \begin{itemize}
              \item UC1.1 Visualizzazione barche
              \item UC1.2 Inserimento barca
              \item UC1.3 Modifica barca
              \item UC1.4 Eliminazione barca
          \end{itemize}
    \item UC2: Calendario uscite \textit{[Astratto]}
          \begin{itemize}
              \item UC2.1 Visualizzazione uscite
              \item UC2.2 Inserimento uscita
              \item UC2.3 Modifica uscita
              \item UC2.4 Eliminazione uscita
          \end{itemize}
    \item UC3: Registrazione utente
    \item UC4: Login utente
\end{itemize}
Dopo averli descritti in maniera testuale si è passati alla realizzazione del diagramma dei componenti e del deployment diagram.
Infine il codice è stato implementato sulla base di questi documenti.

\section{UC1: Gestione barche}

\emph{Breve descrizione}: il proprietario del diving center deve avere una visione generale e un controllo completo delle sue barche all'interno dell'applicazione.
Oltre alla visualizzazione, il proprietario deve poter modificare ed eliminare le imbarcazioni già presenti nel sistema e poter aggiungerne di nuove.
La gestione delle barche è quindi suddivisa in 4 casi d'uso concreti:

\begin{itemize}
    \item UC1.1 Visualizzazione barche
    \item UC1.2 Inserimento barca
    \item UC1.3 Modifica barca
    \item UC1.4 Eliminazione barca
\end{itemize}
~\\
\emph{Attori coinvolti}: Proprietario, Sistema.\\\\
\emph{Trigger}: login del proprietario avvenuto con successo.\\\\
\emph{Postcondizione}: il proprietario dopo il login, viene indirizzato alla home page e può scegliere se cliccare uno dei button: \textit{Visualizza barche} o \textit{Aggiungi barca}. Il primo per visualizzare
tutte le barche inserite, il secondo per aggiungerne una nuova.\\\\
\emph{Procedimento}:

\begin{enumerate}
    \item Proprietario effettua il login.
    \item Sistema mostra la pagina iniziale che comprende un button \textit{Visualizza barche} e un button \textit{Aggiungi barca} con cui il proprietario può concretamente gestire le sue barche.
\end{enumerate}

\section{UC1.1 Visualizzazione barche}

\emph{Breve descrizione:} Il proprietario visualizza le barche inserite nel sistema e le relative caratteristiche.\\\\
\emph{Attori coinvolti:} Proprietario, Sistema.\\\\
\emph{Trigger:} Il proprietario clicca il button \textit{Visualizza barche}.\\\\
\emph{Postcondizione:} Il sistema mostra una vista con l'elenco delle barche.\\\\
\emph{Procedimento}

\begin{enumerate}
    \item Proprietario effettua il login
    \item Proprietario clicca il button \textit{Visualizza barche} presente nella home page.
    \item Sistema mostra l'elenco delle barche presenti nel database con le seguenti informazioni:
          \begin{itemize}
              \item Nome della barca
              \item Modello della barca
              \item Numero posti disponibili sulla barca
              \item Altre caratteristiche TODO
          \end{itemize}
    \item Per ogni item dell'elenco il proprietario può cliccare su \textit{Modifica barca} o \textit{Elimina barca} per modificare o eliminare una barca.
\end{enumerate}

\section{UC1.2 Inserimento barca}

\emph{Breve descrizione}: Il proprietario aggiunge una barca nel sistema. Questo viene fatto principalmente nella fase iniziale in cui il proprietario deve inserire
tutte le sue barche all'interno dell'applicazione e poi in seguito all'acquisto di nuove barche.\\\\
\emph{Attori coinvolti}: Proprietario, Sistema.\\\\
\emph{Trigger}: Il proprietario clicca il button \textit{Inserisci barca}.\\\\
\emph{Postcondizione}: Il sistema mostra la nuova barca all'interno della pagina di visualizzazione delle barche.\\\\
\emph{Procedimento}:

\begin{enumerate}
    \item Il proprietario si trova sulla home page e clicca il button \textit{Inserisci barca}.
    \item Sistema mostra un form in cui è possibile inserire i dati relativi alla barca.
    \item Proprietario riempie il form.
    \item Proprietario clicca su \textit{Conferma} per inserire la barca nel sistema.
    \item Il sistema mostra la nuova barca all'interno della pagina di visualizzazione delle barche.
\end{enumerate}

\section{UC1.3 Modifica barca}

\emph{Breve descrizione}: Il proprietario si trova sulla pagina di visualizzazione delle barche e clicca su un button o un'icona di modifica di una delle barche.
Tale modifica può essere dovuta a cambiamenti strutturali della barca, per esempio la riduzione del numero di posti disponibili.
\\\\
\emph{Attori coinvolti}: Proprietario, Sistema.\\\\
\emph{Trigger}: Il proprietario clicca il button o l'icona \textit{Modifica barca}.\\\\
\emph{Postcondizione}: Il sistema mostra la barca modificata all'interno della pagina di visualizzazione delle barche.\\\\
\emph{Procedimento}:

\begin{enumerate}
    \item Il proprietario si trova sulla pagina di visualizzazione delle barche e clicca su un button o un'icona di modifica di una delle barche.
    \item Sistema mostra una vista in cui è possibile modificare i dati relativi alla barca.
    \item Proprietario modifica i dati.
    \item Proprietario clicca su \textit{Conferma} per modificare i dati della barca.
    \item Il sistema mostra la barca aggiornata all'interno della pagina di visualizzazione delle barche.
\end{enumerate}

\section{UC1.4 Eliminazione barca}

\emph{Breve descrizione}: Il proprietario si trova sulla pagina di visualizzazione delle barche e clicca su un button o un'icona di eliminazione di una delle barche.
L'eliminazione può essere definitiva o temporanea. L'eliminazione è definitiva se la barca non è più agibile o perché viene sostituita da un'altra;
è temporanea nel caso in cui sia necessaria attività di manutenzione.\\\\
\emph{Attori coinvolti}: Proprietario, Sistema.\\\\
\emph{Trigger}: Il proprietario clicca il button o l'icona \textit{Elimina barca}.\\\\
\emph{Postcondizione}: Il sistema mostra la pagina di visualizzazione delle barche in cui non comparirà la barca eliminata.\\\\
\emph{Procedimento}:

\begin{enumerate}
    \item Il proprietario si trova sulla pagina di visualizzazione delle barche e clicca su un button o un'icona di eliminazione di una delle barche. Il proprietario deve inoltre
          scegliere se l'eliminazione è temporanea o definitiva.
    \item Sistema mostra un alert che avvisa il proprietario che l'azione è irreversibile.
    \item Il proprietario può scegliere se confermare l'eliminazione cliccando su \textit{Conferma} o annullare l'azione cliccando su \textit{Annulla}.
    \item Il sistema mostra la pagina di visualizzazione delle barche in cui non comparirà la barca eliminata se il proprietario ha scelto di eliminarla in maniera definitiva.
          Se la barca è stata eliminata temporaneamente sarà sempre visibile nella pagina di visualizzazione delle barche, ma segnalando al proprietario che il sistema organizzerà
          le uscite senza tenere conto di quella barca.
\end{enumerate}

\section{UC2: Calendario delle uscite}
L'applicazione deve poter semplificare la gestione e l'organizzazione delle uscite al proprietario del diving center.
\\Da questo punto di vista l'applicazione deve fungere da calendario, mettendo a disposizione le seguenti funzionalità:

\begin{itemize}
    \item L'inserimento di una data in cui si rende disponibile la prenotazione di uscite, indicando i turni con i relativi orari e durate.
    \item Visualizzazione di tutte le date e turni messi a disposizione.
    \item La modifica o la cancellazione delle date e dei turni, in seguito a cambiamenti climatici o altri imprevisti.
\end{itemize}

\section{UC3: Registrazione dell'utente}
\emph{Breve descrizione}: L'utente compila il form per la registrazione all'app e, se non si è già registrato, viene aggiunto al database.\\\\
\emph{Attori coinvolti}: Utente, Sistema?\\\\
\emph{Trigger}: L'utente preme su "Registrazione"\\\\
\emph{Postcondizione}: L'utente è stato inserito nel database e ha ricevuto la conferma dell'operazione.\\\\
\emph{Procedimento}:
\begin{enumerate}
    \item Utente preme su "Registrazione" nella pagina iniziale dell'app
    \item Utente fornisce Nome, username, indirizzo email e password nel form di registrazione
    \item Il sistema verifica se esiste già un utente con quella mail:
          \begin{enumerate}
              \item se esiste già, il sistema lo comunica all'utente
              \item se non esiste allora il sistema aggiunge l'utente nel database
          \end{enumerate}
\end{enumerate}

\section{UC4: Login dell'utente}
\emph{Breve descrizione}: L'utente (utente normale o il proprietario) compila il form per il login e se le credenziali sono giuste il sistema gli consente l'accesso.\\\\
\emph{Attori coinvolti}: Utente/Proprietario, Sistema?\\\\
\emph{Trigger}: L'utente preme su "Login"\\\\
\emph{Postcondizione}: L'utente ha accesso alla vista del suo profilo.\\\\
\emph{Procedimento}:
\begin{enumerate}
    \item Utente preme su "Login" nella pagina iniziale dell'app
    \item Utente fornisce username e password nel form di registrazione
    \item Il sistema controlla le credenziali inserite:
          \begin{enumerate}
              \item se sono corrette, il sistema invia una conferma di accesso all'utente
              \item se sono errate, il sistema lo comunica all'utente
          \end{enumerate}
\end{enumerate}

\section{UML Component diagram}
I casi d'uso scelti in questa prima iterazione vengono rappresentati sottoforma di componenti nel diagramma in Figura~\ref{fig:componentDiagram}.
Si è scelto di suddividire i caso d'uso in componenti:

\begin{itemize}
    \item <<boundary>> rappresentati dai componenti lato front-end con cui gli attori si interfacciano direttamente. Tali componenti richiedono delle interfacce al back-end.
    \item <<control>> rappresentati dai componenti lato back-end che forniscono delle API al front-end, richiedonone a loro volta al database.
    \item <<data>> rappresentato dal database in cui verranno memorizzati i dati delle barche, il calendario delle uscite e i dati dell'utente in occasione della registrazione.
\end{itemize}

\begin{figure}[h]
    \centering
    \includegraphics[scale=0.27]{ComponentDiagram_v1.png}
    \caption{Diagramma dei componenti UML.}\label{fig:componentDiagram}
\end{figure}
\begin{figure}[h]
    \centering
    \includegraphics[scale=0.27]{ComponentDiagram.png}
    \caption{Diagramma dei componenti UML.}\label{fig:componentDiagramLinda}
\end{figure}

\section{UML Class diagram per Interfacce}
Il seguente diagramma serve a rappresentare le interfacce del sistema e le loro interazioni.
\begin{figure}[h]
    \centering
    \includegraphics[scale=0.27]{ClassDiagramInterfaces.png}
    \caption{Diagramma dei componenti UML.}\label{fig:ClassDiagramInterfaces}
\end{figure}

\section{UML Class diagram per tipi di dato}
Il seguente diagramma serve ad esplicitare tipi di dato particolari e i loro legami.
\begin{figure}[h]
    \centering
    \includegraphics[scale=0.27]{ClassDiagramTypes.png}
    \caption{Diagramma dei componenti UML.}\label{fig:ClassDiagramTypes}
\end{figure}

\section{UML Deployment diagram}
I componenti descritti precedentemente vengono istanziati nel Deployment diagram. In Figura~\ref{fig:deploymentDiagram} vengono mostrati i componenti contenuti nei seguenti nodi:

\begin{itemize}
    \item Cellulare proprietario è il nodo su cui l'admin del sistema potrà gestire le barche e il calendario delle uscite.
    \item Cellulare utente è il nodo su cui una persona potrà registrarsi alla piattaforma.
    \item Web server fornisce le API richieste dall'applicativo lato front-end.
    \item Database funge da storage dei dati.
\end{itemize}

\begin{figure}[h]
    \centering
    \includegraphics[scale=0.4]{DeploymentDiagram_v1.png}
    \caption{Deployment diagram UML.}\label{fig:deploymentDiagram}
\end{figure}
\begin{figure}[h]
    \centering
    \includegraphics[scale=0.4]{DeploymentDiagram.png}
    \caption{Deployment diagram UML.}\label{fig:deploymentDiagramLinda}
\end{figure}